\documentclass{article}


\usepackage{arxiv}

\usepackage[utf8]{inputenc} % allow utf-8 input
\usepackage[T1]{fontenc}    % use 8-bit T1 fonts
\usepackage{hyperref}       % hyperlinks
\usepackage{url}            % simple URL typesetting
\usepackage{booktabs}       % professional-quality tables
\usepackage{amsfonts,amsmath} % blackboard math symbols
\usepackage{nicefrac}       % compact symbols for 1/2, etc.
\usepackage{microtype}      % microtypography
\usepackage[bbgreekl]{mathbbol} % for blackboard \sigma



\renewcommand{\div}{\operatorname{div}}
\newcommand{\eq}[1]{\begin{equation}#1\end{equation}}
\newcommand{\nn}{\vc n}
\newcommand{\uu}{\vc u}
\newcommand{\vc}[1]{\boldsymbol{#1}}
\newcommand{\xx}{\vc x}


\title{Stochastic modeling of EGS using continuum-fracture approach and multilevel Monte Carlo}


\author{
  Jan Březina \\
  \And
  Pavel Exner \\
  \And
  Jan Stebel \\
  \And
  Martin Špetlík \\
  %% \AND
  %% Coauthor \\
  %% Affiliation \\
  %% Address \\
  %% \texttt{email} \\
  %% \And
  %% Coauthor \\
  %% Affiliation \\
  %% Address \\
  %% \texttt{email} \\
  %% \And
  %% Coauthor \\
  %% Affiliation \\
  %% Address \\
  %% \texttt{email} \\
}

\begin{document}
\maketitle

\begin{abstract}

\end{abstract}


% keywords can be removed
\keywords{First keyword \and Second keyword \and More}


\section{Introduction}



\section{Mathematical model}

\subsection{Geological properties}

Rock type: granite, gneiss or mica schist \cite{capova} % zula, rula, svor

Liquid: water

\begin{tabular}{|l|c|c|c|}
\hline
 & rock & fractures & liquid\\\hline
Density $\varrho$ [kg.m${}^{-3}$] & 2700 & 2700 & 1000 \\
Hydraulic conductivity $k$ [m.s${}^{-1}$] & $10^{-9}$ \cite{sperl-trckova} & $10^{-3}$ & ---\\
Porosity $\vartheta$ [--] & $5\times10^{-3}$ \cite{intera} & $5\times10^{-2}$ & --- \\
Compressibility $\beta$ [Pa${}^{-1}$] & $2\times10^{-11}$ \cite{zisman} & $2\times10^{-11}$ & $5\times10^{-10}$ \\
Storativity $S$ [m${}^{-1}$] & $2\times10^{-7}$ & $2\times10^{-7}$ & --- \\
Cross-section $\delta$ [m] & --- & $10^{-3}$ & --- \\
\hline
Young modulus $G$ [Pa] & $50\times10^9$ \cite{ljunggren} & 50 & --- \\
Poisson's ratio $\nu$ [--] & 0.25 \cite{ljunggren} & 0.25 & --- \\
Biot's coefficient $\alpha$ [--] & 1 & 1 & --- \\
\hline
Heat capacity $c$ [J.kg${}^{-1}$.K${}^{-1}$] & 790 & 790 & 4000 \\
Thermal conductivity $\lambda$ [W.m${}^{-1}$.K${}^{-1}$] & 2.5 & 2.5 & 0.5 \\
\hline
\end{tabular}

Storativity is calculated as follows \cite{brace-et-al}:
\[ S = \varrho_l g(\beta_r + \vartheta\beta_l), \]
where $g$ is the standard gravity [m.s${}^{-2}$], which gives approximately the same values both for rock and fractures.

\subsection{Parameters of EGS site and model geometry}

Number of wells: 2 \\
Depth: 5 km \\
Distance: 200 m \\
Well radius: $\rho = 15$ cm \\
Length of open part of wells: $L=60$ m \\

% Pressure difference / water flux: 60 l/s


In our model, we consider pair of vertical wells with the radius $\rho=0.15$ m in the distance $D=200$ m. The open part of the length $L=60$ m is in the depth $5000$ m. The cube $600 \times 600 \times 600$ m around open part is considered as the computing domain. 

{\bf The test geometry:} The wells are connected by three fractures, two of them intersecting the cylinders and one intersecting the first two fractures.

\subsection{Analytical models in the vicinity of the wells}
In order to keep mesh size treatable, we replace the Dirichlet boundary condition for the pressure applied on the open part of the well by the Robin type condition:
%
\eq{\label{eq:H_robin} k\nabla H\cdot\nn = \sigma(H-H_w), }
%
applied on the cut-out cylinders of the radii $R=10$ m. We assume the constant conductivity $k$ in the vicinity of wells. The parameter $\sigma$ is determined from the analytical radially symmetric solution
of the steady Darcy flow problem with a point source:
\[
    h(r) = -\alpha \log (r) + \beta
\]
where $r$ is the distance from the well and $\alpha$, $\beta$ are arbitrary constants.
The flux through the cylinder of the radius $r$ and the length $L$ is the constant
\[  
    Q = -2\pi \rho L k h'(\rho) = 2\pi L k \alpha,
\]
where we have used the Darcy law with the conductivity $k$. Now we use the analytical solution to fit the Robin type boundary condition:
\[
    Q = 2\pi R L \sigma (h(\rho) - h(R)) = 2\pi R L \sigma \alpha \log(R/\rho),
\]
which combined with previous equality determines $\sigma$ as:
\[ 
    \sigma = \frac{k}{R\log(R/\rho)}.
\]

Since the heat transfer is dominated by the convection in the vicinity of the wells the temperature in the well is transported to the surface of the cut-off cylinders or vice verse in the time $t_0 = R/q$ which for the observed volume fluxes $Q\sim 1$ l/s is about $100$ days. This is negligible with respect to the total lifetime of the heat exchanger.


{\bf Original JS idea:}
This flux has to be equal to the flux through the surface of the extended well cylinders:
\[
    Q = 2\pi R l q,\quad q=\sigma(h_\rho - h)
\]
On the surface of the artificial cylinders we shall extrapolate the value of piezometric head as follows: Given value $H_w$ of piezometric head at the well surface (with radius $r_w$) we assume that the piezometric head in the vicinity of the well is the function of the distance $d_w$ to the well axis:
\eq{\label{eq:H_log} H(\xx) = H_w\frac{\log d_w(|\xx|)}{\log r_w}. }
Then, on the surface of the cylinder at radius $r_c$ we impose the Robin condition in the Darcy flow equation:
where $\nn$ stands for the unit outward normal vector.
From \eqref{eq:H_log} and \eqref{eq:H_robin} we obtain the transition coefficient
\[ \sigma = \frac{k}{r_c}\left|\frac{\log r_c}{\log r_w}\right| \]
whose value depends on the hydraulic conductivity.


\subsection{Hydro-mechanical model}
\label{sc:hm_model}

In the first model scenario the fluid is injected to both wells in order to open the preexisting fractures.
The result is the modified cross-section of fractures.
For simplicity we stimulate both wells simultaneously which produces very similar result in the change of fracture cross-section as if the wells were stimulated one by one.
The liquid is injected for 1 day under the pressure 10 MPa (which corresponds to piezometric head $10^3$ m).

Initial condition: zero piezometric head

Boundary conditions:
\begin{itemize}
\item piezometric head $10^3$ m on surface of wells (including the intersections with fractures)
\item piezometric head 0 m on lateral sides of the box
\item no fluid flow through bottom and top part of the box and fracture tips
\item zero displacement on bottom part of the box, zero displacement in the direction orthogonal to box sides
\item zero traction elsewhere
\end{itemize}

We use the Biot poroelasticity equations in the 3d continuum and 2d network of fractures. By dimension reduction of the Biot equations we obtain fluxes and forces which couple the rock with the lower dimensional fractures:
\[ \partial_t(Sp + \alpha\div\uu) - k\Delta p = 0,\qquad -\mu\div(\nabla\uu+\nabla\uu^\top) - \lambda\nabla\div\uu + \alpha\varrho g\nabla p = 0. \]
Here $p=H-x_3$ is the pressure head [m], $\uu$ is the displacement [m], $\mu$ and $\lambda$ are the Lamé parameters [Pa].
...


\subsection{Thermo-hydraulic model}

The second scenario corresponds to operation of the EGS within 30 years after hydraulic stimulation performed according to Sec. \ref{sc:hm_model}.
The fluid is continuously injected into one well and pumped out of the second one under the pressure difference 2 MPa and with inlet temperature $15\ ^\circ$C.

Initial condition:
temperature corresponding to $10\ ^\circ$C at the Earth surface and geothermal gradient $30\ ^\circ$C.km${}^{-1}$

Boundary conditions:
\begin{itemize}
\item piezometric head $\pm10^2$ m on surface of wells ($+$ on the injection well, $-$ on the production well)
\item no flux on the external boundary of the box and on fracture tips
\item temperature $15\ ^\circ$C on surface of injection well (ZDE BY SE MELA UDELAT APROXIMACE DIRICHLETA ROBINEM PODOBNE JAKO V PROUDENI Z NEJAKEHO ANALYTICKEHO RESENI)
\item initial temperature on lateral sides of the box where fluid flows into the box
\item no heat flux through bottom and top part of the box and fracture tips
\end{itemize}

We use the following system of heat equation and steady-state Darcy flow with coupling terms connecting the rock to the fractures:
\[ ... \]


\begin{thebibliography}{9}

\bibitem{brace-et-al} W. F. Brace,  J. B. Walsh, W. T. Grango.  Permeability of Granite Under High Pressure, J. Geophys. Res. 73(6):2225--2236, 1968.

\bibitem{capova} L. Čápová. Specification of the geothermic model in the environs of several selected boreholes. Diploma thesis, Charles University in Prague, 2013.

\bibitem{intera} INTERA Environmental Consultants, Inc. Porosity, Permeability,  and Their  Relationship  in  
Granite,  Basalt, and Tuff. Accession  DE83-011519, NTIS, Springfield, Virginia, 1983.

\bibitem{sperl-trckova} J. Šperl, J. Trčková. Permeability and porosity of rocks and their relationship based on laboratory testing. Acta Geodyn. Geomater. 5(149):41--47, 2008.

\bibitem{ljunggren} C. Ljunggren, O. Stephansson, O. Alm, H. Hakami, U. Mattila. Mechanical properties of granitic rocks from Gide\aa, Sweden. Technical Report 85-06, SKB, 1985.

\bibitem{zisman} W. A. Zisman. Compressibility and anisotropy of rocks at and near the Earth's surface. Proceedings of the National Academy of Sciences of the United States of America, 19(7):666--679, 1933.

\end{thebibliography}

\end{document}
